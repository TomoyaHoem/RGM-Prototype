\documentclass{article}
\usepackage{graphicx}
\usepackage[ngerman]{babel}
\usepackage[utf8]{inputenc}


\begin{document}

\title{\vspace{-2cm}Thesis Proposal - Procedural Generation of Rube-Goldberg Machines}
\author{Tomoya Hoemberg und Alexander Dockhorn}

\maketitle

\section{Zielstellung}

Rube-Goldberg Maschinen erfüllen eine bestimmte Aufgabe absichtlich in zahlreichen unnötigen und komplizierten Einzelschritten. Trotzdessen, dass diese keinen praktischen Nutzen erfüllen, faszinieren sie Menschen weltweit.

Ihre Erstellung geschieht zumeist durch die Hand eines menschlichen Tüftlers, welcher diese aus künstlerischen oder Unterhaltungszwecken erstellt. Im Rahmen der Arbeit soll geprüft werden inwiefern Rube-Goldberg Maschinen durch die Anwendung Evolutionärer Algorithmen generiert und kombiniert werden können und ob diese eine ähnliche Qualität erreichen können.

Im Rahmen der Arbeit sollen daher folgende Fragestellungen beantwortet werden:
\begin{itemize}
	\item Wie können Rube-Goldberg Maschinen kodiert werden?
	\item Welche Genetischen Operatoren sind sinnvoll für die Kombination mehrerer Individuen.
	\item Welche Nebenbedingungen führen zur Generierung von vielfältigen Rube-Goldberg Maschinen.
\end{itemize}

Für die weitere Evaluierung subjektiver Kriterien, z.B. wahrgenommene Vielfalt erstellter Maschinen oder optische Wirking, bietet sich eine Nutzerstudie an.


\section{Methodik}
Themenschwerpunkt der Arbeit ist der Einsatz mehrkriterieller Optimierunsansätze (multi-objective optimization) zur Erstellung von Rube-Goldberg Maschinen unter mehreren Nebenbedingungen. Mögliche Nebenbedingungen sind:
\begin{itemize}
	\item Verwende eine vorgegebene Anzahl an Objekte
	\item Bedecke eine größt/kleinstmögliche Fläche
	\item Maximiere die Anzahl an wechselseitigen Interaktionen zwischen den Objekten
\end{itemize}

Hierbei soll ein Vergleich zwischen a-posteriori und a-priori Methoden erfolgen.
Inhaltliche Schwerpunkte können hierbei sein:
\begin{itemize}
	\item a-posteriori / a-priori multi-objective optimization
	\item Pareto-Optimalität
	\item Pareto-Front Optimierung
\end{itemize}


\section{Aufgabenstellungen}
Die theoretische Analyse der Aufgabenstellung soll mit der Entwicklung eines ausführbaren Prototypens begleitet werden.
Dieses soll folgende Anforderungen erfüllen:

\begin{itemize}
	\item freie Definition von Objekten und ihren Eigenschaften
	\item Platzierung von Objekten durch den Nutzer
	\item Wahl der Nebenbedingungen
	\item im Falle eines a-priori Algorithmus die Einstellung einer Gewichtung der Zielkriterien
	\item im Falle eines a-posteriori Algorithmus die Darstellung einzelner Individuen der Pareto-Front
	\item eine geeignete Visualisierung der genetischen Operatoren
	\item Simulation/Ausführung der Rube-Goldberg Maschine
\end{itemize}

\vfill
\end{document}